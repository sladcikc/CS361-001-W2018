\documentclass[14pt]{article}
\usepackage{cite}
\title{Vision Statement - Walk the Dog: A reminder app}
\author{Camden Sladcik - sladcikc@oregonstate.edu}

\begin{document}
	\maketitle
	\tableofcontents
\newpage	
\section{Overview}
I am a dog owner, and I have a hard time getting out to walk the dog. That is how I came up with the idea for this vision statement.\\
My vision is for a mobile app that can remind you to walk your dog! Among other things, it will keep a log of when you do actually take your dog for a walk. That way you can see if you are getting out for walks as often as you wish you had been.

	 
\subsection{Problem}
Dog owners, do you ever forget to walk your dog? Do you wish you had a good way to be reminded during your busy life?\\
I have a hard time getting out to walk my dog sometimes. Living in Oregon, our weather can add into that struggle. Most dogs need to be exercised everyday. While you could play indoors with your dog to exercise it; getting outside for a walk as gets you some exercise! 

\subsection{Solution}
This is where a mobile app can help dog owners to be reminded or to motivate them to get outside with their dog. There are plenty of mobile apps out there that remind you to do things better. Like eating healthier~\cite{fitnesspal}, or getting in your steps~\cite{stepcounter}. This app will send daily reminders to get outside and walk your dog. If the weather is too miserable for a walk, the reminder will suggest staying indoors and playing with toys instead.\\
A calender section in the app can track which days you do get out for a walk. This will help you know if you are getting out for walks as often as you wanted to. This solution is different from previous due to the inclusion of current/future weather updates included in the reminders.

\subsection{Approach}
As previously stated, this vision will be implemented as a mobile application. More specifically, an android application.\\
The application should include a link into the current location, or have the user set their location. This information will only be used to draw current and future weather information to be displayed in app and notifications. A calender will be in the app, past days will include information on whether a walk occurred, and future days will include weather data. \\
Notifications will need to be pushed. These notifications will be giving a user defined time to go off each day. However, if the weather may not be to the user's liking (rain etc), the notification should be displayed earlier than the user specified time. If plus or minus one hour from the user defined reminder time has a better weather outlook, the app shall suggest walking during this time instead. If no time is decent, the app shall suggest playing indoors.

\subsection{Risks}
This approach has little limitations. The most limiting factor will be the developers familiarity with producing android applications. I believe linking current location data and receiving up-to-date hourly weather data is the most technical aspect of this project

\bibliography{myref}
\bibliographystyle{plain}

\end{document}